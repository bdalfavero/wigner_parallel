\documentclass{article}

\title{The Truncated Wigner method on an HPC cluster}
\author{Benjamin DalFavero}
\date{\today}

\usepackage{physics}

\newcommand{\Schroedinger}{Schr{\"o}dinger}

\begin{document}

\maketitle

\section{Introduction} \label{introduction}

Systems of many interacting, bosonic particles are of great interest in the areas of condensed matter physics, 
chemsitry, quantum sensing, and quantum analogues. At thermal equilibrium, these systems can be treated well by 
Markov chain Monte Carlo (MCMC) methods. For systems out of equilibrium, the path integral Monte Carlo approach 
is infeasible due to the famous ``Monte Carlo sign problem,'' where fast fluctuations in the integrand require
an enormous number of samples for convergence. Lagrangian approaches using the path integral must thus be abandoned
in favor of Hamiltonian approaches in the \Schroedinger{} or Heisenberg pictures. With these methods, the most 
fundamental challenge is the exponential scaling of the Hilbert space with the size of the system.

This paper outlines the implementation of the MCMC methods for Truncated Wigner approximation on 
an HPC cluster, with analysis of how performance is improved by using various constructs from distributed
computing. Section \ref{theory-methods} outlines the structure of the theory, as well as MCMC methods for 
these problems. Section \ref{parallel} details the technical aspects of performing the MCMC calculations 
on the HPC system. Section \ref{results} presents an analysis of the speedup provided by these parallel computing 
techniques. Section \ref{conclusion} concludes the paper, presenting an overview of the work and areas of future 
improvement.

\section{Theory and numerical methods} \label{theory-methods}

The methods in this paper use the phase-space formulation to estimate the seminclassical dynamics 
of the Bose-Hubbard model.

\subsection{The Bose-Hubbard model}

The Bose-Hubbard model (BHM) is a foundational model in condensed matter and atomic physics. 
The model consists of an \(n\)-dimensional lattice with \(M\) sites. The lattice sites 
are occupied by \(N\) interacting bosons. A configuration of this system is described by a Fock state
\begin{equation}
    \label{equ:fock-state}
    \ket{n_1 n_2 \ldots n_M}
\end{equation}
where each \(n_i\) is the number of bosons at lattice site \(i\). The state of the quantum system is a 
superposition of these Fock states:
\begin{equation}
    \label{equ:wave-function}
    \ket{\Psi} = \sum_{n_1, \ldots, n_M} c_{n_1 \ldots n_M} \ket{n_1 n_2 \ldots n_M},
\end{equation}
where the coefficients \(c_{n_1 \ldots n_M}\) are complex numbers. 

Linear maps on quantum states of the form in Eq. \ref{equ:wave-function} are generating 
using creation and annihilation operators
\begin{equation}
    b_i^\dagger \ket{n_1 n_2 \ldots n_i \ldots n_M} = \sqrt{n_i + 1} \ket{n_1 n_2 \ldots (n_i + 1) \ldots n_M}
\end{equation}
\begin{equation}
    b_i \ket{n_1 n_2 \ldots n_i \ldots n_M} = \sqrt{n_i} \ket{n_1 n_2 \ldots (n_i - 1) \ldots n_M}
\end{equation}
which map Fock states to other Fock states. The eigenstates of the annihilation operator 
are the coherent states, which have the relation
\begin{equation}
    \label{equ:coherent-state}
    b_i \ket{\beta_1 \ldots \beta_i \ldots \beta_M} = \beta_i \ket{\beta_1 \ldots \beta_i \ldots \beta_M}
\end{equation}
where the variables \(\beta_i\) are complex numbers. 

The dynamics of the system are governed by the \Schroedinger{} equation 
\begin{equation}
    \label{equ:schroedinger}
    i \hbar \frac{\partial}{\partial t} \ket{\Psi(t)} = \hat{H}_{BH} \ket{\Psi(t)},
\end{equation}
where \(\hbar\) is the reduced Planck constant, \(\ket{\Psi}(t)\) is the state of the system at time \(t\),
and \(\hat{H}_{BH}\) is the Bose-Hubbard Hamiltonian:
\begin{equation}
    \hat{H}_{BH} = -t \sum_{i,j \in NN(i)} (b_i^\dagger b_j + \textrm{h.c.}) + g \sum_i b_i^\dagger b_i^\dagger b_i b_i,
\end{equation}
where \(i,j\) denote lattice sites, \(NN(i)\) is the set of nearest neighbors for lattice site \(i\), \(t\) is the hopping
energy which determines the kinetic energy of a boson hopping from site \(i\) to site \(j\), and \(g\) is the strength 
of on-site repulsive interactions.

\subsection{The Truncated Wigner approximation}

The Trucated Wigner approximation (TWA) is a semiclassical approximation of the dynamics of a bosonic system 
given by the \Schroedinger equation. The TWA relies of Wigner's phase space formalism of quantum mechanics to 
build approximations. 

In the phase space formulation, the quantum state in Eq. \ref{equ:wave-function} is instead represented by the 
Wigner function:
\begin{equation}
    \label{equ:wigner-function}
    W(\psi_i, \psi^\ast_i) = \int d\eta d\eta^\ast \bra{\psi - \eta/2} \hat{\rho} \ket{\psi + \eta/2} 
    \exp(-|\psi|^2 - \frac{1}{4}|\eta|^2) \exp(\frac{1}{2} \eta^\ast \psi - \eta \psi^\ast)
\end{equation}
where \(\psi\) and \(\eta\) are vectors of complex variables, and \(\rho\) is the density matrix. 

\subsection{Solution by Monte Carlo methods}

\section{Parallel implementation} \label{parallel}

Since this method involves both random sampling and solving systems of ordinary differential equations, it is 
imperative to parallelize over both types of tasks. 
There are two convenient ways to divide the workload between nodes such a calculation:
\begin{enumerate}
    \item Each node in the cluster is responsible for a certain number of Monte Carlo samples, 
    and it holds the entire ``wave function'' vector for each sample. After each time step, 
    the nodes average their populations together. 
    \item Each node handles a part of the domain, but does the calculation for every sample 
    on the same node. Each time step, the nodes need to share their boundary points with 
    each other to enable time-stepping. 
\end{enumerate}
In an average calculation, we expect that the number of MCMC samples will vastly outnumber of the number of lattice sites. 
So, I have chosen the first option.

Secondly, we must decide how the time-stepping and population averaging will be done on the single nodes. 

\section{Results} \label{results}

\subsection{Convergence and verification}

\subsection{Speedup from distributed-memory parallelism}

\subsection{Speedup from shared-memory parallelism}

\section{Conclusion} \label{conclusion}

\end{document}